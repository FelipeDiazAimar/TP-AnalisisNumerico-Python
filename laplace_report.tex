
\documentclass[12pt]{article}
\usepackage[utf8]{inputenc}
\usepackage[T1]{fontenc}
\usepackage{amsmath,amssymb}
\usepackage[spanish]{babel}
\usepackage{hyperref}
	\title{TP1 - Transformadas de Laplace}
\author{Nombre: Felipe Diaz Aimar \\Legajo: 16841 \\Materia: Análisis Numérico \\Comisión: Comisión K}
\date{\today}

\begin{document}
\maketitle

\section*{Ejercicio 1}
Sea $F(t) = \left(4 t + 3\right)^{2}$. Su transformada de Laplace es
\[
\mathcal{L}[F(t)](s) = \frac{9}{s} + \frac{24}{s^{2}} + \frac{32}{s^{3}}
\]

\section*{Ejercicio 2}
Sea $f(t) = \begin{cases} e^{t}, & 0 \le t < 1 \\ 0, & 1 \le t < \tfrac{3}{2} \\ 1, & t > \tfrac{3}{2} \end{cases}$. Su transformada de Laplace es
\[
\mathcal{L}[f(t)](s) = \frac{1 - e^{1 - s}}{s - 1} + \frac{e^{- \frac{3 s}{2}}}{s}
\]

\section*{Compilación alternativa (Overleaf)}
Si no cuenta con latexmk/MiKTeX/TeX Live en su equipo, puede compilar este informe en línea:
\begin{enumerate}
    \item Visite \href{https://www.overleaf.com}{Overleaf}.
    \item Cree un proyecto en blanco (Blank Project).
    \item Copie y pegue el contenido de este archivo (laplace\_report.tex) en el editor de Overleaf y presione compilar.
\end{enumerate}
\end{document}
